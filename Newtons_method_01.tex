
% Default to the notebook output style

    


% Inherit from the specified cell style.




    
\documentclass[11pt]{article}

    
    
    \usepackage[T1]{fontenc}
    % Nicer default font (+ math font) than Computer Modern for most use cases
    \usepackage{mathpazo}

    % Basic figure setup, for now with no caption control since it's done
    % automatically by Pandoc (which extracts ![](path) syntax from Markdown).
    \usepackage{graphicx}
    % We will generate all images so they have a width \maxwidth. This means
    % that they will get their normal width if they fit onto the page, but
    % are scaled down if they would overflow the margins.
    \makeatletter
    \def\maxwidth{\ifdim\Gin@nat@width>\linewidth\linewidth
    \else\Gin@nat@width\fi}
    \makeatother
    \let\Oldincludegraphics\includegraphics
    % Set max figure width to be 80% of text width, for now hardcoded.
    \renewcommand{\includegraphics}[1]{\Oldincludegraphics[width=.8\maxwidth]{#1}}
    % Ensure that by default, figures have no caption (until we provide a
    % proper Figure object with a Caption API and a way to capture that
    % in the conversion process - todo).
    \usepackage{caption}
    \DeclareCaptionLabelFormat{nolabel}{}
    \captionsetup{labelformat=nolabel}

    \usepackage{adjustbox} % Used to constrain images to a maximum size 
    \usepackage{xcolor} % Allow colors to be defined
    \usepackage{enumerate} % Needed for markdown enumerations to work
    \usepackage{geometry} % Used to adjust the document margins
    \usepackage{amsmath} % Equations
    \usepackage{amssymb} % Equations
    \usepackage{textcomp} % defines textquotesingle
    % Hack from http://tex.stackexchange.com/a/47451/13684:
    \AtBeginDocument{%
        \def\PYZsq{\textquotesingle}% Upright quotes in Pygmentized code
    }
    \usepackage{upquote} % Upright quotes for verbatim code
    \usepackage{eurosym} % defines \euro
    \usepackage[mathletters]{ucs} % Extended unicode (utf-8) support
    \usepackage[utf8x]{inputenc} % Allow utf-8 characters in the tex document
    \usepackage{fancyvrb} % verbatim replacement that allows latex
    \usepackage{grffile} % extends the file name processing of package graphics 
                         % to support a larger range 
    % The hyperref package gives us a pdf with properly built
    % internal navigation ('pdf bookmarks' for the table of contents,
    % internal cross-reference links, web links for URLs, etc.)
    \usepackage{hyperref}
    \usepackage{longtable} % longtable support required by pandoc >1.10
    \usepackage{booktabs}  % table support for pandoc > 1.12.2
    \usepackage[inline]{enumitem} % IRkernel/repr support (it uses the enumerate* environment)
    \usepackage[normalem]{ulem} % ulem is needed to support strikethroughs (\sout)
                                % normalem makes italics be italics, not underlines
    \usepackage{mathrsfs}
    

    
    
    % Colors for the hyperref package
    \definecolor{urlcolor}{rgb}{0,.145,.698}
    \definecolor{linkcolor}{rgb}{.71,0.21,0.01}
    \definecolor{citecolor}{rgb}{.12,.54,.11}

    % ANSI colors
    \definecolor{ansi-black}{HTML}{3E424D}
    \definecolor{ansi-black-intense}{HTML}{282C36}
    \definecolor{ansi-red}{HTML}{E75C58}
    \definecolor{ansi-red-intense}{HTML}{B22B31}
    \definecolor{ansi-green}{HTML}{00A250}
    \definecolor{ansi-green-intense}{HTML}{007427}
    \definecolor{ansi-yellow}{HTML}{DDB62B}
    \definecolor{ansi-yellow-intense}{HTML}{B27D12}
    \definecolor{ansi-blue}{HTML}{208FFB}
    \definecolor{ansi-blue-intense}{HTML}{0065CA}
    \definecolor{ansi-magenta}{HTML}{D160C4}
    \definecolor{ansi-magenta-intense}{HTML}{A03196}
    \definecolor{ansi-cyan}{HTML}{60C6C8}
    \definecolor{ansi-cyan-intense}{HTML}{258F8F}
    \definecolor{ansi-white}{HTML}{C5C1B4}
    \definecolor{ansi-white-intense}{HTML}{A1A6B2}
    \definecolor{ansi-default-inverse-fg}{HTML}{FFFFFF}
    \definecolor{ansi-default-inverse-bg}{HTML}{000000}

    % commands and environments needed by pandoc snippets
    % extracted from the output of `pandoc -s`
    \providecommand{\tightlist}{%
      \setlength{\itemsep}{0pt}\setlength{\parskip}{0pt}}
    \DefineVerbatimEnvironment{Highlighting}{Verbatim}{commandchars=\\\{\}}
    % Add ',fontsize=\small' for more characters per line
    \newenvironment{Shaded}{}{}
    \newcommand{\KeywordTok}[1]{\textcolor[rgb]{0.00,0.44,0.13}{\textbf{{#1}}}}
    \newcommand{\DataTypeTok}[1]{\textcolor[rgb]{0.56,0.13,0.00}{{#1}}}
    \newcommand{\DecValTok}[1]{\textcolor[rgb]{0.25,0.63,0.44}{{#1}}}
    \newcommand{\BaseNTok}[1]{\textcolor[rgb]{0.25,0.63,0.44}{{#1}}}
    \newcommand{\FloatTok}[1]{\textcolor[rgb]{0.25,0.63,0.44}{{#1}}}
    \newcommand{\CharTok}[1]{\textcolor[rgb]{0.25,0.44,0.63}{{#1}}}
    \newcommand{\StringTok}[1]{\textcolor[rgb]{0.25,0.44,0.63}{{#1}}}
    \newcommand{\CommentTok}[1]{\textcolor[rgb]{0.38,0.63,0.69}{\textit{{#1}}}}
    \newcommand{\OtherTok}[1]{\textcolor[rgb]{0.00,0.44,0.13}{{#1}}}
    \newcommand{\AlertTok}[1]{\textcolor[rgb]{1.00,0.00,0.00}{\textbf{{#1}}}}
    \newcommand{\FunctionTok}[1]{\textcolor[rgb]{0.02,0.16,0.49}{{#1}}}
    \newcommand{\RegionMarkerTok}[1]{{#1}}
    \newcommand{\ErrorTok}[1]{\textcolor[rgb]{1.00,0.00,0.00}{\textbf{{#1}}}}
    \newcommand{\NormalTok}[1]{{#1}}
    
    % Additional commands for more recent versions of Pandoc
    \newcommand{\ConstantTok}[1]{\textcolor[rgb]{0.53,0.00,0.00}{{#1}}}
    \newcommand{\SpecialCharTok}[1]{\textcolor[rgb]{0.25,0.44,0.63}{{#1}}}
    \newcommand{\VerbatimStringTok}[1]{\textcolor[rgb]{0.25,0.44,0.63}{{#1}}}
    \newcommand{\SpecialStringTok}[1]{\textcolor[rgb]{0.73,0.40,0.53}{{#1}}}
    \newcommand{\ImportTok}[1]{{#1}}
    \newcommand{\DocumentationTok}[1]{\textcolor[rgb]{0.73,0.13,0.13}{\textit{{#1}}}}
    \newcommand{\AnnotationTok}[1]{\textcolor[rgb]{0.38,0.63,0.69}{\textbf{\textit{{#1}}}}}
    \newcommand{\CommentVarTok}[1]{\textcolor[rgb]{0.38,0.63,0.69}{\textbf{\textit{{#1}}}}}
    \newcommand{\VariableTok}[1]{\textcolor[rgb]{0.10,0.09,0.49}{{#1}}}
    \newcommand{\ControlFlowTok}[1]{\textcolor[rgb]{0.00,0.44,0.13}{\textbf{{#1}}}}
    \newcommand{\OperatorTok}[1]{\textcolor[rgb]{0.40,0.40,0.40}{{#1}}}
    \newcommand{\BuiltInTok}[1]{{#1}}
    \newcommand{\ExtensionTok}[1]{{#1}}
    \newcommand{\PreprocessorTok}[1]{\textcolor[rgb]{0.74,0.48,0.00}{{#1}}}
    \newcommand{\AttributeTok}[1]{\textcolor[rgb]{0.49,0.56,0.16}{{#1}}}
    \newcommand{\InformationTok}[1]{\textcolor[rgb]{0.38,0.63,0.69}{\textbf{\textit{{#1}}}}}
    \newcommand{\WarningTok}[1]{\textcolor[rgb]{0.38,0.63,0.69}{\textbf{\textit{{#1}}}}}
    
    
    % Define a nice break command that doesn't care if a line doesn't already
    % exist.
    \def\br{\hspace*{\fill} \\* }
    % Math Jax compatibility definitions
    \def\gt{>}
    \def\lt{<}
    \let\Oldtex\TeX
    \let\Oldlatex\LaTeX
    \renewcommand{\TeX}{\textrm{\Oldtex}}
    \renewcommand{\LaTeX}{\textrm{\Oldlatex}}
    % Document parameters
    % Document title
    \title{Newtons\_method\_01}
    
    
    
    
    

    % Pygments definitions
    
\makeatletter
\def\PY@reset{\let\PY@it=\relax \let\PY@bf=\relax%
    \let\PY@ul=\relax \let\PY@tc=\relax%
    \let\PY@bc=\relax \let\PY@ff=\relax}
\def\PY@tok#1{\csname PY@tok@#1\endcsname}
\def\PY@toks#1+{\ifx\relax#1\empty\else%
    \PY@tok{#1}\expandafter\PY@toks\fi}
\def\PY@do#1{\PY@bc{\PY@tc{\PY@ul{%
    \PY@it{\PY@bf{\PY@ff{#1}}}}}}}
\def\PY#1#2{\PY@reset\PY@toks#1+\relax+\PY@do{#2}}

\expandafter\def\csname PY@tok@w\endcsname{\def\PY@tc##1{\textcolor[rgb]{0.73,0.73,0.73}{##1}}}
\expandafter\def\csname PY@tok@c\endcsname{\let\PY@it=\textit\def\PY@tc##1{\textcolor[rgb]{0.25,0.50,0.50}{##1}}}
\expandafter\def\csname PY@tok@cp\endcsname{\def\PY@tc##1{\textcolor[rgb]{0.74,0.48,0.00}{##1}}}
\expandafter\def\csname PY@tok@k\endcsname{\let\PY@bf=\textbf\def\PY@tc##1{\textcolor[rgb]{0.00,0.50,0.00}{##1}}}
\expandafter\def\csname PY@tok@kp\endcsname{\def\PY@tc##1{\textcolor[rgb]{0.00,0.50,0.00}{##1}}}
\expandafter\def\csname PY@tok@kt\endcsname{\def\PY@tc##1{\textcolor[rgb]{0.69,0.00,0.25}{##1}}}
\expandafter\def\csname PY@tok@o\endcsname{\def\PY@tc##1{\textcolor[rgb]{0.40,0.40,0.40}{##1}}}
\expandafter\def\csname PY@tok@ow\endcsname{\let\PY@bf=\textbf\def\PY@tc##1{\textcolor[rgb]{0.67,0.13,1.00}{##1}}}
\expandafter\def\csname PY@tok@nb\endcsname{\def\PY@tc##1{\textcolor[rgb]{0.00,0.50,0.00}{##1}}}
\expandafter\def\csname PY@tok@nf\endcsname{\def\PY@tc##1{\textcolor[rgb]{0.00,0.00,1.00}{##1}}}
\expandafter\def\csname PY@tok@nc\endcsname{\let\PY@bf=\textbf\def\PY@tc##1{\textcolor[rgb]{0.00,0.00,1.00}{##1}}}
\expandafter\def\csname PY@tok@nn\endcsname{\let\PY@bf=\textbf\def\PY@tc##1{\textcolor[rgb]{0.00,0.00,1.00}{##1}}}
\expandafter\def\csname PY@tok@ne\endcsname{\let\PY@bf=\textbf\def\PY@tc##1{\textcolor[rgb]{0.82,0.25,0.23}{##1}}}
\expandafter\def\csname PY@tok@nv\endcsname{\def\PY@tc##1{\textcolor[rgb]{0.10,0.09,0.49}{##1}}}
\expandafter\def\csname PY@tok@no\endcsname{\def\PY@tc##1{\textcolor[rgb]{0.53,0.00,0.00}{##1}}}
\expandafter\def\csname PY@tok@nl\endcsname{\def\PY@tc##1{\textcolor[rgb]{0.63,0.63,0.00}{##1}}}
\expandafter\def\csname PY@tok@ni\endcsname{\let\PY@bf=\textbf\def\PY@tc##1{\textcolor[rgb]{0.60,0.60,0.60}{##1}}}
\expandafter\def\csname PY@tok@na\endcsname{\def\PY@tc##1{\textcolor[rgb]{0.49,0.56,0.16}{##1}}}
\expandafter\def\csname PY@tok@nt\endcsname{\let\PY@bf=\textbf\def\PY@tc##1{\textcolor[rgb]{0.00,0.50,0.00}{##1}}}
\expandafter\def\csname PY@tok@nd\endcsname{\def\PY@tc##1{\textcolor[rgb]{0.67,0.13,1.00}{##1}}}
\expandafter\def\csname PY@tok@s\endcsname{\def\PY@tc##1{\textcolor[rgb]{0.73,0.13,0.13}{##1}}}
\expandafter\def\csname PY@tok@sd\endcsname{\let\PY@it=\textit\def\PY@tc##1{\textcolor[rgb]{0.73,0.13,0.13}{##1}}}
\expandafter\def\csname PY@tok@si\endcsname{\let\PY@bf=\textbf\def\PY@tc##1{\textcolor[rgb]{0.73,0.40,0.53}{##1}}}
\expandafter\def\csname PY@tok@se\endcsname{\let\PY@bf=\textbf\def\PY@tc##1{\textcolor[rgb]{0.73,0.40,0.13}{##1}}}
\expandafter\def\csname PY@tok@sr\endcsname{\def\PY@tc##1{\textcolor[rgb]{0.73,0.40,0.53}{##1}}}
\expandafter\def\csname PY@tok@ss\endcsname{\def\PY@tc##1{\textcolor[rgb]{0.10,0.09,0.49}{##1}}}
\expandafter\def\csname PY@tok@sx\endcsname{\def\PY@tc##1{\textcolor[rgb]{0.00,0.50,0.00}{##1}}}
\expandafter\def\csname PY@tok@m\endcsname{\def\PY@tc##1{\textcolor[rgb]{0.40,0.40,0.40}{##1}}}
\expandafter\def\csname PY@tok@gh\endcsname{\let\PY@bf=\textbf\def\PY@tc##1{\textcolor[rgb]{0.00,0.00,0.50}{##1}}}
\expandafter\def\csname PY@tok@gu\endcsname{\let\PY@bf=\textbf\def\PY@tc##1{\textcolor[rgb]{0.50,0.00,0.50}{##1}}}
\expandafter\def\csname PY@tok@gd\endcsname{\def\PY@tc##1{\textcolor[rgb]{0.63,0.00,0.00}{##1}}}
\expandafter\def\csname PY@tok@gi\endcsname{\def\PY@tc##1{\textcolor[rgb]{0.00,0.63,0.00}{##1}}}
\expandafter\def\csname PY@tok@gr\endcsname{\def\PY@tc##1{\textcolor[rgb]{1.00,0.00,0.00}{##1}}}
\expandafter\def\csname PY@tok@ge\endcsname{\let\PY@it=\textit}
\expandafter\def\csname PY@tok@gs\endcsname{\let\PY@bf=\textbf}
\expandafter\def\csname PY@tok@gp\endcsname{\let\PY@bf=\textbf\def\PY@tc##1{\textcolor[rgb]{0.00,0.00,0.50}{##1}}}
\expandafter\def\csname PY@tok@go\endcsname{\def\PY@tc##1{\textcolor[rgb]{0.53,0.53,0.53}{##1}}}
\expandafter\def\csname PY@tok@gt\endcsname{\def\PY@tc##1{\textcolor[rgb]{0.00,0.27,0.87}{##1}}}
\expandafter\def\csname PY@tok@err\endcsname{\def\PY@bc##1{\setlength{\fboxsep}{0pt}\fcolorbox[rgb]{1.00,0.00,0.00}{1,1,1}{\strut ##1}}}
\expandafter\def\csname PY@tok@kc\endcsname{\let\PY@bf=\textbf\def\PY@tc##1{\textcolor[rgb]{0.00,0.50,0.00}{##1}}}
\expandafter\def\csname PY@tok@kd\endcsname{\let\PY@bf=\textbf\def\PY@tc##1{\textcolor[rgb]{0.00,0.50,0.00}{##1}}}
\expandafter\def\csname PY@tok@kn\endcsname{\let\PY@bf=\textbf\def\PY@tc##1{\textcolor[rgb]{0.00,0.50,0.00}{##1}}}
\expandafter\def\csname PY@tok@kr\endcsname{\let\PY@bf=\textbf\def\PY@tc##1{\textcolor[rgb]{0.00,0.50,0.00}{##1}}}
\expandafter\def\csname PY@tok@bp\endcsname{\def\PY@tc##1{\textcolor[rgb]{0.00,0.50,0.00}{##1}}}
\expandafter\def\csname PY@tok@fm\endcsname{\def\PY@tc##1{\textcolor[rgb]{0.00,0.00,1.00}{##1}}}
\expandafter\def\csname PY@tok@vc\endcsname{\def\PY@tc##1{\textcolor[rgb]{0.10,0.09,0.49}{##1}}}
\expandafter\def\csname PY@tok@vg\endcsname{\def\PY@tc##1{\textcolor[rgb]{0.10,0.09,0.49}{##1}}}
\expandafter\def\csname PY@tok@vi\endcsname{\def\PY@tc##1{\textcolor[rgb]{0.10,0.09,0.49}{##1}}}
\expandafter\def\csname PY@tok@vm\endcsname{\def\PY@tc##1{\textcolor[rgb]{0.10,0.09,0.49}{##1}}}
\expandafter\def\csname PY@tok@sa\endcsname{\def\PY@tc##1{\textcolor[rgb]{0.73,0.13,0.13}{##1}}}
\expandafter\def\csname PY@tok@sb\endcsname{\def\PY@tc##1{\textcolor[rgb]{0.73,0.13,0.13}{##1}}}
\expandafter\def\csname PY@tok@sc\endcsname{\def\PY@tc##1{\textcolor[rgb]{0.73,0.13,0.13}{##1}}}
\expandafter\def\csname PY@tok@dl\endcsname{\def\PY@tc##1{\textcolor[rgb]{0.73,0.13,0.13}{##1}}}
\expandafter\def\csname PY@tok@s2\endcsname{\def\PY@tc##1{\textcolor[rgb]{0.73,0.13,0.13}{##1}}}
\expandafter\def\csname PY@tok@sh\endcsname{\def\PY@tc##1{\textcolor[rgb]{0.73,0.13,0.13}{##1}}}
\expandafter\def\csname PY@tok@s1\endcsname{\def\PY@tc##1{\textcolor[rgb]{0.73,0.13,0.13}{##1}}}
\expandafter\def\csname PY@tok@mb\endcsname{\def\PY@tc##1{\textcolor[rgb]{0.40,0.40,0.40}{##1}}}
\expandafter\def\csname PY@tok@mf\endcsname{\def\PY@tc##1{\textcolor[rgb]{0.40,0.40,0.40}{##1}}}
\expandafter\def\csname PY@tok@mh\endcsname{\def\PY@tc##1{\textcolor[rgb]{0.40,0.40,0.40}{##1}}}
\expandafter\def\csname PY@tok@mi\endcsname{\def\PY@tc##1{\textcolor[rgb]{0.40,0.40,0.40}{##1}}}
\expandafter\def\csname PY@tok@il\endcsname{\def\PY@tc##1{\textcolor[rgb]{0.40,0.40,0.40}{##1}}}
\expandafter\def\csname PY@tok@mo\endcsname{\def\PY@tc##1{\textcolor[rgb]{0.40,0.40,0.40}{##1}}}
\expandafter\def\csname PY@tok@ch\endcsname{\let\PY@it=\textit\def\PY@tc##1{\textcolor[rgb]{0.25,0.50,0.50}{##1}}}
\expandafter\def\csname PY@tok@cm\endcsname{\let\PY@it=\textit\def\PY@tc##1{\textcolor[rgb]{0.25,0.50,0.50}{##1}}}
\expandafter\def\csname PY@tok@cpf\endcsname{\let\PY@it=\textit\def\PY@tc##1{\textcolor[rgb]{0.25,0.50,0.50}{##1}}}
\expandafter\def\csname PY@tok@c1\endcsname{\let\PY@it=\textit\def\PY@tc##1{\textcolor[rgb]{0.25,0.50,0.50}{##1}}}
\expandafter\def\csname PY@tok@cs\endcsname{\let\PY@it=\textit\def\PY@tc##1{\textcolor[rgb]{0.25,0.50,0.50}{##1}}}

\def\PYZbs{\char`\\}
\def\PYZus{\char`\_}
\def\PYZob{\char`\{}
\def\PYZcb{\char`\}}
\def\PYZca{\char`\^}
\def\PYZam{\char`\&}
\def\PYZlt{\char`\<}
\def\PYZgt{\char`\>}
\def\PYZsh{\char`\#}
\def\PYZpc{\char`\%}
\def\PYZdl{\char`\$}
\def\PYZhy{\char`\-}
\def\PYZsq{\char`\'}
\def\PYZdq{\char`\"}
\def\PYZti{\char`\~}
% for compatibility with earlier versions
\def\PYZat{@}
\def\PYZlb{[}
\def\PYZrb{]}
\makeatother


    % Exact colors from NB
    \definecolor{incolor}{rgb}{0.0, 0.0, 0.5}
    \definecolor{outcolor}{rgb}{0.545, 0.0, 0.0}



    
    % Prevent overflowing lines due to hard-to-break entities
    \sloppy 
    % Setup hyperref package
    \hypersetup{
      breaklinks=true,  % so long urls are correctly broken across lines
      colorlinks=true,
      urlcolor=urlcolor,
      linkcolor=linkcolor,
      citecolor=citecolor,
      }
    % Slightly bigger margins than the latex defaults
    
    \geometry{verbose,tmargin=1in,bmargin=1in,lmargin=1in,rmargin=1in}
    
    

    \begin{document}
    
    
    \maketitle
    
    

    
    \section{Bisection Method}\label{bisection-method}

This program can be used to find the roots of a function using the
Bisection Method. The user must define a function and visually inspect
the graph for the location of the roots, or zeroes. Then, by entering
two x values, one 'above' and one 'below' the suspected root, the
program will run through n iterations to find the root within the
desired accuracy. For an indepth description of the method, visit:
\href{http://www.sosmath.com/calculus/limcon/limcon07/limcon07.html}{SOS
Math: The Bisection Method}

    \begin{Verbatim}[commandchars=\\\{\}]
{\color{incolor}In [{\color{incolor}1}]:} \PY{c+c1}{\PYZsh{}We import some prepackaged math and graphing stuff}
        
        \PY{k+kn}{import} \PY{n+nn}{matplotlib}\PY{n+nn}{.}\PY{n+nn}{pyplot} \PY{k}{as} \PY{n+nn}{plt}
        \PY{k+kn}{from} \PY{n+nn}{math} \PY{k}{import} \PY{o}{*}
        \PY{k+kn}{import} \PY{n+nn}{numpy} \PY{k}{as} \PY{n+nn}{np}
\end{Verbatim}

    \subsection{The function whose roots we want to
investigate}\label{the-function-whose-roots-we-want-to-investigate}

\[V_{eff}=\frac{1}{r^{2}}-\frac{1}{r}=-0.15\]

    \begin{Verbatim}[commandchars=\\\{\}]
{\color{incolor}In [{\color{incolor}2}]:} \PY{c+c1}{\PYZsh{}The function that we will investigate has to be defined}
        \PY{k}{def} \PY{n+nf}{g}\PY{p}{(}\PY{n}{x}\PY{p}{)}\PY{p}{:}
            \PY{n}{y} \PY{o}{=} \PY{p}{(}\PY{l+m+mi}{1} \PY{o}{/}\PY{p}{(} \PY{n}{x} \PY{o}{*}\PY{o}{*} \PY{l+m+mi}{2}\PY{p}{)}\PY{p}{)} \PY{o}{\PYZhy{}} \PY{p}{(}\PY{l+m+mi}{1} \PY{o}{/} \PY{n}{x}\PY{p}{)} \PY{o}{+} \PY{o}{.}\PY{l+m+mi}{15}
            \PY{k}{return} \PY{n}{y}
\end{Verbatim}

    \begin{Verbatim}[commandchars=\\\{\}]
{\color{incolor}In [{\color{incolor}3}]:} \PY{c+c1}{\PYZsh{}Here, we utilize the pyplot package to graph the function defined above}
        \PY{c+c1}{\PYZsh{}The domain, x, can be defined using linspace(start, end, increments)}
        \PY{n}{x} \PY{o}{=} \PY{n}{np}\PY{o}{.}\PY{n}{linspace}\PY{p}{(}\PY{o}{.}\PY{l+m+mi}{7}\PY{p}{,} \PY{l+m+mf}{6.7}\PY{p}{,} \PY{l+m+mi}{100}\PY{p}{)}
        \PY{c+c1}{\PYZsh{}Zero function (y=0) added so the x\PYZhy{}axis is clearly visible}
        \PY{n}{zero} \PY{o}{=} \PY{n}{np}\PY{o}{.}\PY{n}{linspace}\PY{p}{(}\PY{l+m+mi}{0} \PY{p}{,}\PY{l+m+mi}{0} \PY{p}{,}\PY{l+m+mi}{100}\PY{p}{)}
        \PY{n}{plt}\PY{o}{.}\PY{n}{plot}\PY{p}{(}\PY{n}{x}\PY{p}{,} \PY{n}{g}\PY{p}{(}\PY{n}{x}\PY{p}{)}\PY{p}{)}
        \PY{n}{plt}\PY{o}{.}\PY{n}{plot}\PY{p}{(}\PY{n}{x}\PY{p}{,} \PY{n}{zero}\PY{p}{)}
        \PY{n}{plt}\PY{o}{.}\PY{n}{grid}\PY{p}{(}\PY{k+kc}{True}\PY{p}{)}
        \PY{n}{plt}\PY{o}{.}\PY{n}{title}\PY{p}{(}\PY{l+s+s1}{\PYZsq{}}\PY{l+s+s1}{V\PYZus{}eff = 1/r\PYZca{}2 \PYZhy{} 1/r = \PYZhy{}0.15}\PY{l+s+s1}{\PYZsq{}}\PY{p}{,} \PY{n}{fontsize}\PY{o}{=}\PY{l+m+mi}{16}\PY{p}{)}
        \PY{n}{plt}\PY{o}{.}\PY{n}{xlabel}\PY{p}{(}\PY{l+s+s1}{\PYZsq{}}\PY{l+s+s1}{x}\PY{l+s+s1}{\PYZsq{}}\PY{p}{,} \PY{n}{fontsize}\PY{o}{=}\PY{l+m+mi}{14}\PY{p}{)}
        \PY{n}{plt}\PY{o}{.}\PY{n}{ylabel}\PY{p}{(}\PY{l+s+s1}{\PYZsq{}}\PY{l+s+s1}{f(x)}\PY{l+s+s1}{\PYZsq{}}\PY{p}{,} \PY{n}{fontsize}\PY{o}{=}\PY{l+m+mi}{14}\PY{p}{)}
        \PY{n}{plt}\PY{o}{.}\PY{n}{show}\PY{p}{(}\PY{p}{)}
\end{Verbatim}

    \begin{center}
    \adjustimage{max size={0.9\linewidth}{0.9\paperheight}}{output_4_0.png}
    \end{center}
    { \hspace*{\fill} \\}
    
    \subsection{This is the main loop of our
program}\label{this-is-the-main-loop-of-our-program}

These next blocks execute the loop that:

\begin{itemize}
\item
  computes the function values for x values near the root
\item
  if the product is + then use the midpoint as the new lower value
\item
  if the product is - then use the midpoint as the new upper value
\item
  At each iteration print the midpoint and the error
\end{itemize}

After 10 loops (more or less depending on required accuracy) print the
final result. The error is generated at each iteration using the
standard equation, (another way to run the loop is to specify a value
for \(\epsilon\), say .00001, and the loop can be executed until that
tolerance is met):

    \[\begin{align*}\epsilon &= \text{error for each loop} \\
 n &= \text{# of iterations} \\
 c_{n} &= n^{th}\text{ midpoint} \\
 c &= \text{actual value of root} \\
\epsilon &= \left | c_{n} - c \right | \leq \frac{x_{2} - x_{1}}{2^{n}}
\end{align*}\]

    \subsubsection{Root \# 1}\label{root-1}

From the graph, we can clearly see that the function crosses the x-axis
at two points. One root looks to be around 1.25 so, we will use x=0.5
and x=1.5 as our surrounding values.

    \begin{Verbatim}[commandchars=\\\{\}]
{\color{incolor}In [{\color{incolor}4}]:} \PY{n}{x\PYZus{}1} \PY{o}{=} \PY{l+m+mf}{1.0} \PY{c+c1}{\PYZsh{}Input a value less than the expected root}
        \PY{n}{x\PYZus{}2} \PY{o}{=} \PY{l+m+mf}{1.5} \PY{c+c1}{\PYZsh{}Input a value greater than the expected root}
                  \PY{c+c1}{\PYZsh{}(but less than the next root)}
        \PY{n}{counter} \PY{o}{=} \PY{l+m+mi}{0} \PY{c+c1}{\PYZsh{}We also initialize a counter to keep track of the iterations}
        \PY{k}{for} \PY{n}{i} \PY{o+ow}{in} \PY{n+nb}{range}\PY{p}{(}\PY{l+m+mi}{10}\PY{p}{)}\PY{p}{:}
            \PY{n}{x\PYZus{}mid} \PY{o}{=} \PY{p}{(}\PY{n}{x\PYZus{}1} \PY{o}{+} \PY{n}{x\PYZus{}2}\PY{p}{)} \PY{o}{/} \PY{l+m+mi}{2}
            \PY{n}{test} \PY{o}{=} \PY{n}{g}\PY{p}{(}\PY{n}{x\PYZus{}1}\PY{p}{)} \PY{o}{*} \PY{n}{g}\PY{p}{(}\PY{n}{x\PYZus{}mid}\PY{p}{)}
            \PY{k}{if} \PY{n}{test} \PY{o}{\PYZgt{}} \PY{l+m+mi}{0}\PY{p}{:}
                \PY{n}{x\PYZus{}1} \PY{o}{=} \PY{n}{x\PYZus{}mid}
            \PY{k}{else}\PY{p}{:}
                \PY{n}{x\PYZus{}2} \PY{o}{=} \PY{n}{x\PYZus{}mid}
            \PY{n}{counter} \PY{o}{+}\PY{o}{=} \PY{l+m+mi}{1}
            \PY{n}{error} \PY{o}{=} \PY{n+nb}{abs}\PY{p}{(}\PY{p}{(}\PY{n}{x\PYZus{}2} \PY{o}{\PYZhy{}} \PY{n}{x\PYZus{}1}\PY{p}{)}\PY{p}{)} \PY{o}{/} \PY{l+m+mi}{2} \PY{o}{*}\PY{o}{*} \PY{n}{counter}
            \PY{n+nb}{print}\PY{p}{(}\PY{l+s+s1}{\PYZsq{}}\PY{l+s+si}{\PYZob{}:\PYZlt{}15\PYZcb{}}\PY{l+s+s1}{ : }\PY{l+s+si}{\PYZob{}:\PYZlt{}15\PYZcb{}}\PY{l+s+s1}{\PYZsq{}} \PY{o}{.}\PY{n}{format}\PY{p}{(}\PY{n}{x\PYZus{}mid}\PY{p}{,} \PY{n}{error}\PY{p}{)}\PY{p}{)}
        \PY{n}{root\PYZus{}1} \PY{o}{=} \PY{n}{x\PYZus{}mid}    
        \PY{n+nb}{print}\PY{p}{(}\PY{l+s+s2}{\PYZdq{}}\PY{l+s+se}{\PYZbs{}n}\PY{l+s+s2}{A root for this function is x = }\PY{l+s+s2}{\PYZdq{}} \PY{l+s+s1}{\PYZsq{}}\PY{l+s+si}{\PYZpc{}.7f}\PY{l+s+s1}{\PYZsq{}} \PY{o}{\PYZpc{}} \PY{n}{x\PYZus{}mid}\PY{p}{)}
        \PY{n+nb}{print}\PY{p}{(}\PY{l+s+s2}{\PYZdq{}}\PY{l+s+s2}{The error after }\PY{l+s+si}{\PYZob{}\PYZcb{}}\PY{l+s+s2}{ iterations is }\PY{l+s+si}{\PYZob{}:.7f\PYZcb{}}\PY{l+s+s2}{\PYZdq{}}\PY{o}{.}\PY{n}{format}\PY{p}{(}\PY{n}{counter}\PY{p}{,} \PY{n}{error}\PY{p}{)}\PY{p}{)}
\end{Verbatim}

    \begin{Verbatim}[commandchars=\\\{\}]
1.25            : 0.125          
1.125           : 0.03125        
1.1875          : 0.0078125      
1.21875         : 0.001953125    
1.234375        : 0.00048828125  
1.2265625       : 0.0001220703125
1.22265625      : 3.0517578125e-05
1.224609375     : 7.62939453125e-06
1.2255859375    : 1.9073486328125e-06
1.22509765625   : 4.76837158203125e-07

A root for this function is x = 1.2250977
The error after 10 iterations is 0.0000005

    \end{Verbatim}

    \subsubsection{Root \# 2}\label{root-2}

The same loop can be reused with a different range surrounding the next
root. Since this root appears to be around x=5.5 we can set our starting
values as x1=5.0 and x2=6.0.

    \begin{Verbatim}[commandchars=\\\{\}]
{\color{incolor}In [{\color{incolor}5}]:} \PY{n}{x\PYZus{}1} \PY{o}{=} \PY{l+m+mf}{5.0}
        \PY{n}{x\PYZus{}2} \PY{o}{=} \PY{l+m+mf}{6.0}
        \PY{n}{counter} \PY{o}{=} \PY{l+m+mi}{0}
        \PY{k}{for} \PY{n}{i} \PY{o+ow}{in} \PY{n+nb}{range}\PY{p}{(}\PY{l+m+mi}{10}\PY{p}{)}\PY{p}{:}
            \PY{n}{x\PYZus{}mid} \PY{o}{=} \PY{p}{(}\PY{n}{x\PYZus{}1} \PY{o}{+} \PY{n}{x\PYZus{}2}\PY{p}{)} \PY{o}{/} \PY{l+m+mi}{2}
            \PY{n}{test} \PY{o}{=} \PY{n}{g}\PY{p}{(}\PY{n}{x\PYZus{}1}\PY{p}{)} \PY{o}{*} \PY{n}{g}\PY{p}{(}\PY{n}{x\PYZus{}mid}\PY{p}{)}
            \PY{k}{if} \PY{n}{test} \PY{o}{\PYZgt{}} \PY{l+m+mi}{0}\PY{p}{:}
                \PY{n}{x\PYZus{}1} \PY{o}{=} \PY{n}{x\PYZus{}mid}
            \PY{k}{else}\PY{p}{:}
                \PY{n}{x\PYZus{}2} \PY{o}{=} \PY{n}{x\PYZus{}mid}
            \PY{n}{counter} \PY{o}{+}\PY{o}{=} \PY{l+m+mi}{1}
            \PY{n}{error} \PY{o}{=} \PY{n+nb}{abs}\PY{p}{(}\PY{p}{(}\PY{n}{x\PYZus{}2} \PY{o}{\PYZhy{}} \PY{n}{x\PYZus{}1}\PY{p}{)}\PY{p}{)} \PY{o}{/} \PY{l+m+mi}{2} \PY{o}{*}\PY{o}{*} \PY{n}{counter}
            \PY{n+nb}{print}\PY{p}{(}\PY{l+s+s1}{\PYZsq{}}\PY{l+s+si}{\PYZob{}:\PYZlt{}15\PYZcb{}}\PY{l+s+s1}{ : }\PY{l+s+si}{\PYZob{}:\PYZlt{}15\PYZcb{}}\PY{l+s+s1}{\PYZsq{}} \PY{o}{.}\PY{n}{format}\PY{p}{(}\PY{n}{x\PYZus{}mid}\PY{p}{,} \PY{n}{error}\PY{p}{)}\PY{p}{)}
        \PY{n}{root\PYZus{}2} \PY{o}{=} \PY{n}{x\PYZus{}mid}   
        \PY{n+nb}{print}\PY{p}{(}\PY{l+s+s2}{\PYZdq{}}\PY{l+s+se}{\PYZbs{}n}\PY{l+s+s2}{A root for this function is x = }\PY{l+s+s2}{\PYZdq{}} \PY{l+s+s1}{\PYZsq{}}\PY{l+s+si}{\PYZpc{}.7f}\PY{l+s+s1}{\PYZsq{}} \PY{o}{\PYZpc{}} \PY{n}{x\PYZus{}mid}\PY{p}{)}
        \PY{n+nb}{print}\PY{p}{(}\PY{l+s+s2}{\PYZdq{}}\PY{l+s+s2}{The error after }\PY{l+s+si}{\PYZob{}\PYZcb{}}\PY{l+s+s2}{ iterations is }\PY{l+s+si}{\PYZob{}:.7f\PYZcb{}}\PY{l+s+s2}{\PYZdq{}}\PY{o}{.}\PY{n}{format}\PY{p}{(}\PY{n}{counter}\PY{p}{,} \PY{n}{error}\PY{p}{)}\PY{p}{)}
\end{Verbatim}

    \begin{Verbatim}[commandchars=\\\{\}]
5.5             : 0.25           
5.25            : 0.0625         
5.375           : 0.015625       
5.4375          : 0.00390625     
5.46875         : 0.0009765625   
5.453125        : 0.000244140625 
5.4453125       : 6.103515625e-05
5.44140625      : 1.52587890625e-05
5.443359375     : 3.814697265625e-06
5.4423828125    : 9.5367431640625e-07

A root for this function is x = 5.4423828
The error after 10 iterations is 0.0000010

    \end{Verbatim}

    \subsection{Conclusion}\label{conclusion}

We have demonstrated that a simple program can quickly return a decimal
approximation of the roots of otherwise intractable functions. For our
effective potential energy function, we were able to to visually
identify the two roots of interest and refine the value to within
.000001.

If this is something we will be doing a lot of or for equations with
multiple roots, we should probably define a function that takes an
equation, graphs it and then allows the user to input two x values in
order to find the root.

\textbf{Once again, the roots of our function are:}

1.22559 and 5.44238

    \subsubsection{Just for fun}\label{just-for-fun}

Just for fun, we run the first loop again, but this time for 25
iterations to see how it performs.

    \begin{Verbatim}[commandchars=\\\{\}]
{\color{incolor}In [{\color{incolor}6}]:} \PY{n}{x\PYZus{}1} \PY{o}{=} \PY{l+m+mf}{1.0} 
        \PY{n}{x\PYZus{}2} \PY{o}{=} \PY{l+m+mf}{1.5} 
        \PY{n}{counter} \PY{o}{=} \PY{l+m+mi}{0} 
        \PY{k}{for} \PY{n}{i} \PY{o+ow}{in} \PY{n+nb}{range}\PY{p}{(}\PY{l+m+mi}{25}\PY{p}{)}\PY{p}{:}
            \PY{n}{x\PYZus{}mid} \PY{o}{=} \PY{p}{(}\PY{n}{x\PYZus{}1} \PY{o}{+} \PY{n}{x\PYZus{}2}\PY{p}{)} \PY{o}{/} \PY{l+m+mi}{2}
            \PY{n}{test} \PY{o}{=} \PY{n}{g}\PY{p}{(}\PY{n}{x\PYZus{}1}\PY{p}{)} \PY{o}{*} \PY{n}{g}\PY{p}{(}\PY{n}{x\PYZus{}mid}\PY{p}{)}
            \PY{k}{if} \PY{n}{test} \PY{o}{\PYZgt{}} \PY{l+m+mi}{0}\PY{p}{:}
                \PY{n}{x\PYZus{}1} \PY{o}{=} \PY{n}{x\PYZus{}mid}
            \PY{k}{else}\PY{p}{:}
                \PY{n}{x\PYZus{}2} \PY{o}{=} \PY{n}{x\PYZus{}mid}
            \PY{n}{counter} \PY{o}{+}\PY{o}{=} \PY{l+m+mi}{1}
            \PY{n}{error} \PY{o}{=} \PY{n+nb}{abs}\PY{p}{(}\PY{p}{(}\PY{n}{x\PYZus{}2} \PY{o}{\PYZhy{}} \PY{n}{x\PYZus{}1}\PY{p}{)}\PY{p}{)} \PY{o}{/} \PY{l+m+mi}{2} \PY{o}{*}\PY{o}{*} \PY{n}{counter}
            \PY{n+nb}{print}\PY{p}{(}\PY{l+s+s1}{\PYZsq{}}\PY{l+s+si}{\PYZob{}:\PYZlt{}20\PYZcb{}}\PY{l+s+s1}{ : }\PY{l+s+si}{\PYZob{}:\PYZlt{}20\PYZcb{}}\PY{l+s+s1}{\PYZsq{}} \PY{o}{.}\PY{n}{format}\PY{p}{(}\PY{n}{x\PYZus{}mid}\PY{p}{,} \PY{n}{error}\PY{p}{)}\PY{p}{)}
        \PY{n}{root\PYZus{}1} \PY{o}{=} \PY{n}{x\PYZus{}mid}    
        \PY{n+nb}{print}\PY{p}{(}\PY{l+s+s2}{\PYZdq{}}\PY{l+s+se}{\PYZbs{}n}\PY{l+s+s2}{A root for this function is x = }\PY{l+s+s2}{\PYZdq{}} \PY{l+s+s1}{\PYZsq{}}\PY{l+s+si}{\PYZpc{}.7f}\PY{l+s+s1}{\PYZsq{}} \PY{o}{\PYZpc{}} \PY{n}{x\PYZus{}mid}\PY{p}{)}
        \PY{n+nb}{print}\PY{p}{(}\PY{l+s+s2}{\PYZdq{}}\PY{l+s+s2}{The error after }\PY{l+s+si}{\PYZob{}\PYZcb{}}\PY{l+s+s2}{ iterations is }\PY{l+s+si}{\PYZob{}:.7f\PYZcb{}}\PY{l+s+s2}{\PYZdq{}}\PY{o}{.}\PY{n}{format}\PY{p}{(}\PY{n}{counter}\PY{p}{,} \PY{n}{error}\PY{p}{)}\PY{p}{)}
\end{Verbatim}

    \begin{Verbatim}[commandchars=\\\{\}]
1.25                 : 0.125               
1.125                : 0.03125             
1.1875               : 0.0078125           
1.21875              : 0.001953125         
1.234375             : 0.00048828125       
1.2265625            : 0.0001220703125     
1.22265625           : 3.0517578125e-05    
1.224609375          : 7.62939453125e-06   
1.2255859375         : 1.9073486328125e-06 
1.22509765625        : 4.76837158203125e-07
1.225341796875       : 1.1920928955078125e-07
1.2252197265625      : 2.9802322387695312e-08
1.22515869140625     : 7.450580596923828e-09
1.225128173828125    : 1.862645149230957e-09
1.2251434326171875   : 4.656612873077393e-10
1.2251510620117188   : 1.1641532182693481e-10
1.2251472473144531   : 2.9103830456733704e-11
1.225149154663086    : 7.275957614183426e-12
1.2251482009887695   : 1.8189894035458565e-12
1.2251486778259277   : 4.547473508864641e-13
1.2251484394073486   : 1.1368683772161603e-13
1.225148320198059    : 2.842170943040401e-14
1.2251482605934143   : 7.105427357601002e-15
1.225148230791092    : 1.7763568394002505e-15
1.2251482158899307   : 4.440892098500626e-16

A root for this function is x = 1.2251482
The error after 25 iterations is 0.0000000

    \end{Verbatim}

    \begin{Verbatim}[commandchars=\\\{\}]
{\color{incolor}In [{\color{incolor} }]:} 
\end{Verbatim}


    % Add a bibliography block to the postdoc
    
    
    
    \end{document}
